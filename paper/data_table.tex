\documentclass[a4paper,landscape]{article}
    \usepackage[landscape]{geometry}
% better looking tables with `\toprule`,`\midrule`,`\bottomrule`:
\usepackage{booktabs}
\begin{document}
\begin{table}[h!]
\begin{center}
\begin{tabular}{lccccc}
\toprule
\textbf{Variant} & \textbf{\# curated sequences} & \textbf{\# automatically extracted sequences} & \textbf{\# features} & \textbf{Taxonomic span} \\
\toprule 
        canonical H3 & 26 &1606  & 11 & Eukaryotes \\
        cenH3 & 14 &276  & 15 & Eukaryotes \\
        H3.3 & 17 &541  & 12 & Eukaryotes \\
        H3.5 & 2 &1135  & 11 & Hominids \\
        H3.Y & 8 &89  & 12 & Primates \\
        TS H3.4 & 2 &2110  & 11 & Mammals \\
\toprule 
        canonical H4 & 14 &7498  & 10 & Eukaryotes \\
\toprule 
        canonical H2A & 39 &4096  & 17 & Eukaryotes \\
        H2A.1 & 2 &846  & 17 & Mammals \\
        H2A.B & 15 &139  & 21 & Mammals \\
        H2A.L & 17 &186  & 17 & Certain mammals \\
        H2A.P & 11 &95  & 17 & Placentalia \\
        H2A.W & 9 &870  & 19 & Plants \\
        H2A.X & 22 &1142  & 19 & Eukaryotes except nematode \\
        H2A.Z & 25 &2609  & 20 & Eukaryotes \\
        macroH2A & 10 &1215  & 19 & Vertebrates(?) \\
\toprule 
        canonical H2B & 27 &6633  & 9 & Eukaryotes \\
        H2B.1 & 4 &443  & 9 & Mammals \\
        H2B.W & 6 &245  & 10 & Mammals \\
        H2B.Z & 3 &208  & 9 & Apicomplexa \\
        sperm H2B & 5 &56  & 10 & Echinoidea(?) \\
        subH2B & 11 &86  & 11 & Primates, rodents, marsupials, and bovids \\
\toprule 
        generic H1 & 18 &4340  & 7 & Eukaryotes \\
        H1.0 & 15 &681  & 7 & Metazoa \\
        H1.10 & 6 &146  & 7 & Vertebrates \\
        OO H1.8 & 2 &250  & 7 & Mammals \\
        scH1 & 2 &404  & 13 & Saccharomyces(?) \\
        TS H1.6 & 8 &474  & 7 & Mammals \\
        TS H1.7 & 2 &144  & 7 & Mammals \\
        TS H1.9 & 4 &101  & 7 & Mammals \\
\toprule 
\bottomrule
\end{tabular}
\end{center}
\end{table}
\end{document}
