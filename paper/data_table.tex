\documentclass[a4paper,landscape]{article}
    \usepackage[landscape]{geometry}
% better looking tables with `\toprule`,`\midrule`,`\bottomrule`:
\usepackage{booktabs}
\begin{document}
\begin{table}[h!]
\begin{center}
\begin{tabular}{lccccc}
\toprule
\textbf{Variant} & \textbf{Curated Sequences} & \textbf{Auto Sequences} & \textbf{Num Features} & \textbf{Taxonomic Span} \\
\toprule 
\toprule 
        genericH1 & 18 &18  & 7 & Eukaryotes \\
        H1.0 & 16 &16  & 7 & Metazoa \\
        H1.10 & 6 &6  & 7 & Vertebrates \\
        OO H1.8 & 2 &2  & 7 & Mammals \\
        ScH1 & 2 &2  & 13 & Saccharomyces(?) \\
        TS H1.6 & 8 &8  & 7 & Mammals \\
        TS H1.7 & 2 &2  & 7 & Mammals \\
        TS H1.9 & 4 &4  & 7 & Mammals \\
\toprule 
        canonicalH2B & 27 &27  & 9 & Eukaryotes \\
        H2B.W & 6 &6  & 10 & Mammals \\
        H2B.Z & 3 &3  & 9 & Apicomplexa \\
        sperm H2B & 6 &6  & 9 & Echinacea(?) \\
        subH2B & 11 &11  & 11 & Primates, rodents, marsupials, and bovids \\
        TS H2B.1 & 4 &4  & 9 & Mammals \\
\toprule 
        canonicalH2A & 37 &37  & 17 & Eukaryotes \\
        H2A.B & 15 &15  & 21 & Mammals \\
        H2A.L & 17 &17  & 17 & Certain mammals \\
        H2A.M & 11 &11  & 17 & Mammals \\
        H2A.W & 10 &11  & 19 & Plants \\
        H2A.X & 23 &24  & 19 & Eukaryotes except nematode \\
        H2A.Z & 26 &27  & 20 & Eukaryotes \\
        macroH2A & 10 &10  & 19 & Vertebrates \\
        TS H2A.1 & 2 &2  & 17 & Mammals \\
\toprule 
        canonicalH4 & 14 &14  & 10 & Eukaryotes \\
\toprule 
        canonicalH3 & 23 &23  & 11 & Eukaryotes \\
        cenH3 & 12 &12  & 14 & Eukaryotes \\
        H3.3 & 15 &15  & 12 & Eukaryotes \\
        H3.5 & 2 &2  & 11 & Hominids \\
        H3.Y & 8 &8  & 12 & Primates \\
        TS H3.4 & 2 &3  & 11 & mammals \\
\bottomrule
\end{tabular}
\end{center}
\end{table}
\end{document}
